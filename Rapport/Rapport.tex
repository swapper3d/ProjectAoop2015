\documentclass{article}
\renewcommand{\textfraction}{0.0}
\usepackage{fancyhdr}
\usepackage{graphicx}
\usepackage{adjustbox}
\usepackage{cite}
\usepackage[hidelinks]{hyperref}
\usepackage[utf8]{inputenc}
%\usepackage[swedish]{babel}
%\usepackage[T1]{fontenc}
%\usepackage[utf8]{inputenc}
%\usepackage[latin1]{inputenc}
%\usepackage[swedish]{babel}
%\pagestyle{fancy}
\begin{document}
\rfoot{}
\lhead{}
\chead{}
\rhead{}
\lfoot{}
%\cfoot{}
\title{Projekt AOOP}
\author{Erik Halvarsson}
\maketitle
\begin{center}
%\includegraphics[scale=0.85]{stake}
\end{center}
\thispagestyle{empty}
\newpage
\section{Introduction}

\newpage
\tableofcontents
\newpage

\section{compiler}

\subsubsection{netjavabeanj}

\begin{itemize}
\item test \cite{java}
\end{itemize}


\subsection{memberof}

\section{javac}

\subsection{indexof}

\begin{figure}[h]
\centerline{
%\includegraphics[scale=0.17]{sketch}
}
\caption{En bild här kanske? \label{fig:sketch}}
\end{figure}

\begin{thebibliography}{9}
\bibitem{java}
	java e skit
\end{thebibliography}


\end{document}